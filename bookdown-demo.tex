\documentclass[]{book}
\usepackage{lmodern}
\usepackage{amssymb,amsmath}
\usepackage{ifxetex,ifluatex}
\usepackage{fixltx2e} % provides \textsubscript
\ifnum 0\ifxetex 1\fi\ifluatex 1\fi=0 % if pdftex
  \usepackage[T1]{fontenc}
  \usepackage[utf8]{inputenc}
\else % if luatex or xelatex
  \ifxetex
    \usepackage{mathspec}
  \else
    \usepackage{fontspec}
  \fi
  \defaultfontfeatures{Ligatures=TeX,Scale=MatchLowercase}
\fi
% use upquote if available, for straight quotes in verbatim environments
\IfFileExists{upquote.sty}{\usepackage{upquote}}{}
% use microtype if available
\IfFileExists{microtype.sty}{%
\usepackage{microtype}
\UseMicrotypeSet[protrusion]{basicmath} % disable protrusion for tt fonts
}{}
\usepackage[margin=1in]{geometry}
\usepackage{hyperref}
\hypersetup{unicode=true,
            pdftitle={A DS Book Example},
            pdfauthor={Jon Minton},
            pdfborder={0 0 0},
            breaklinks=true}
\urlstyle{same}  % don't use monospace font for urls
\usepackage{natbib}
\bibliographystyle{apalike}
\usepackage{color}
\usepackage{fancyvrb}
\newcommand{\VerbBar}{|}
\newcommand{\VERB}{\Verb[commandchars=\\\{\}]}
\DefineVerbatimEnvironment{Highlighting}{Verbatim}{commandchars=\\\{\}}
% Add ',fontsize=\small' for more characters per line
\usepackage{framed}
\definecolor{shadecolor}{RGB}{248,248,248}
\newenvironment{Shaded}{\begin{snugshade}}{\end{snugshade}}
\newcommand{\KeywordTok}[1]{\textcolor[rgb]{0.13,0.29,0.53}{\textbf{#1}}}
\newcommand{\DataTypeTok}[1]{\textcolor[rgb]{0.13,0.29,0.53}{#1}}
\newcommand{\DecValTok}[1]{\textcolor[rgb]{0.00,0.00,0.81}{#1}}
\newcommand{\BaseNTok}[1]{\textcolor[rgb]{0.00,0.00,0.81}{#1}}
\newcommand{\FloatTok}[1]{\textcolor[rgb]{0.00,0.00,0.81}{#1}}
\newcommand{\ConstantTok}[1]{\textcolor[rgb]{0.00,0.00,0.00}{#1}}
\newcommand{\CharTok}[1]{\textcolor[rgb]{0.31,0.60,0.02}{#1}}
\newcommand{\SpecialCharTok}[1]{\textcolor[rgb]{0.00,0.00,0.00}{#1}}
\newcommand{\StringTok}[1]{\textcolor[rgb]{0.31,0.60,0.02}{#1}}
\newcommand{\VerbatimStringTok}[1]{\textcolor[rgb]{0.31,0.60,0.02}{#1}}
\newcommand{\SpecialStringTok}[1]{\textcolor[rgb]{0.31,0.60,0.02}{#1}}
\newcommand{\ImportTok}[1]{#1}
\newcommand{\CommentTok}[1]{\textcolor[rgb]{0.56,0.35,0.01}{\textit{#1}}}
\newcommand{\DocumentationTok}[1]{\textcolor[rgb]{0.56,0.35,0.01}{\textbf{\textit{#1}}}}
\newcommand{\AnnotationTok}[1]{\textcolor[rgb]{0.56,0.35,0.01}{\textbf{\textit{#1}}}}
\newcommand{\CommentVarTok}[1]{\textcolor[rgb]{0.56,0.35,0.01}{\textbf{\textit{#1}}}}
\newcommand{\OtherTok}[1]{\textcolor[rgb]{0.56,0.35,0.01}{#1}}
\newcommand{\FunctionTok}[1]{\textcolor[rgb]{0.00,0.00,0.00}{#1}}
\newcommand{\VariableTok}[1]{\textcolor[rgb]{0.00,0.00,0.00}{#1}}
\newcommand{\ControlFlowTok}[1]{\textcolor[rgb]{0.13,0.29,0.53}{\textbf{#1}}}
\newcommand{\OperatorTok}[1]{\textcolor[rgb]{0.81,0.36,0.00}{\textbf{#1}}}
\newcommand{\BuiltInTok}[1]{#1}
\newcommand{\ExtensionTok}[1]{#1}
\newcommand{\PreprocessorTok}[1]{\textcolor[rgb]{0.56,0.35,0.01}{\textit{#1}}}
\newcommand{\AttributeTok}[1]{\textcolor[rgb]{0.77,0.63,0.00}{#1}}
\newcommand{\RegionMarkerTok}[1]{#1}
\newcommand{\InformationTok}[1]{\textcolor[rgb]{0.56,0.35,0.01}{\textbf{\textit{#1}}}}
\newcommand{\WarningTok}[1]{\textcolor[rgb]{0.56,0.35,0.01}{\textbf{\textit{#1}}}}
\newcommand{\AlertTok}[1]{\textcolor[rgb]{0.94,0.16,0.16}{#1}}
\newcommand{\ErrorTok}[1]{\textcolor[rgb]{0.64,0.00,0.00}{\textbf{#1}}}
\newcommand{\NormalTok}[1]{#1}
\usepackage{longtable,booktabs}
\usepackage{graphicx,grffile}
\makeatletter
\def\maxwidth{\ifdim\Gin@nat@width>\linewidth\linewidth\else\Gin@nat@width\fi}
\def\maxheight{\ifdim\Gin@nat@height>\textheight\textheight\else\Gin@nat@height\fi}
\makeatother
% Scale images if necessary, so that they will not overflow the page
% margins by default, and it is still possible to overwrite the defaults
% using explicit options in \includegraphics[width, height, ...]{}
\setkeys{Gin}{width=\maxwidth,height=\maxheight,keepaspectratio}
\IfFileExists{parskip.sty}{%
\usepackage{parskip}
}{% else
\setlength{\parindent}{0pt}
\setlength{\parskip}{6pt plus 2pt minus 1pt}
}
\setlength{\emergencystretch}{3em}  % prevent overfull lines
\providecommand{\tightlist}{%
  \setlength{\itemsep}{0pt}\setlength{\parskip}{0pt}}
\setcounter{secnumdepth}{5}
% Redefines (sub)paragraphs to behave more like sections
\ifx\paragraph\undefined\else
\let\oldparagraph\paragraph
\renewcommand{\paragraph}[1]{\oldparagraph{#1}\mbox{}}
\fi
\ifx\subparagraph\undefined\else
\let\oldsubparagraph\subparagraph
\renewcommand{\subparagraph}[1]{\oldsubparagraph{#1}\mbox{}}
\fi

%%% Use protect on footnotes to avoid problems with footnotes in titles
\let\rmarkdownfootnote\footnote%
\def\footnote{\protect\rmarkdownfootnote}

%%% Change title format to be more compact
\usepackage{titling}

% Create subtitle command for use in maketitle
\newcommand{\subtitle}[1]{
  \posttitle{
    \begin{center}\large#1\end{center}
    }
}

\setlength{\droptitle}{-2em}

  \title{A DS Book Example}
    \pretitle{\vspace{\droptitle}\centering\huge}
  \posttitle{\par}
    \author{Jon Minton}
    \preauthor{\centering\large\emph}
  \postauthor{\par}
      \predate{\centering\large\emph}
  \postdate{\par}
    \date{2018-07-07}

\usepackage{booktabs}
\usepackage{amsthm}
\makeatletter
\def\thm@space@setup{%
  \thm@preskip=8pt plus 2pt minus 4pt
  \thm@postskip=\thm@preskip
}
\makeatother

\usepackage{amsthm}
\newtheorem{theorem}{Theorem}[chapter]
\newtheorem{lemma}{Lemma}[chapter]
\theoremstyle{definition}
\newtheorem{definition}{Definition}[chapter]
\newtheorem{corollary}{Corollary}[chapter]
\newtheorem{proposition}{Proposition}[chapter]
\theoremstyle{definition}
\newtheorem{example}{Example}[chapter]
\theoremstyle{definition}
\newtheorem{exercise}{Exercise}[chapter]
\theoremstyle{remark}
\newtheorem*{remark}{Remark}
\newtheorem*{solution}{Solution}
\begin{document}
\maketitle

{
\setcounter{tocdepth}{1}
\tableofcontents
}
\chapter{Prerequisites}\label{prerequisites}

This is a \emph{sample} book written in \textbf{Markdown}. You can use
anything that Pandoc's Markdown supports, e.g., a math equation
\(a^2 + b^2 = c^2\).

The \textbf{bookdown} package can be installed from CRAN or Github:

\begin{Shaded}
\begin{Highlighting}[]
\KeywordTok{install.packages}\NormalTok{(}\StringTok{"bookdown"}\NormalTok{)}
\CommentTok{# or the development version}
\CommentTok{#devtools::install_github("rstudio/bookdown")}
\end{Highlighting}
\end{Shaded}

Remember each Rmd file contains one and only one chapter, and a chapter
is defined by the first-level heading \texttt{\#}.

To compile this example to PDF, you need XeLaTeX. You are recommended to
install TinyTeX (which includes XeLaTeX):
\url{https://yihui.name/tinytex/}.

\chapter{Introduction}\label{intro}

You can label chapter and section titles using \texttt{\{\#label\}}
after them, e.g., we can reference Chapter \ref{intro}. If you do not
manually label them, there will be automatic labels anyway, e.g.,
Chapter \ref{methods}.

Figures and tables with captions will be placed in \texttt{figure} and
\texttt{table} environments, respectively.

\begin{Shaded}
\begin{Highlighting}[]
\KeywordTok{par}\NormalTok{(}\DataTypeTok{mar =} \KeywordTok{c}\NormalTok{(}\DecValTok{4}\NormalTok{, }\DecValTok{4}\NormalTok{, .}\DecValTok{1}\NormalTok{, .}\DecValTok{1}\NormalTok{))}
\KeywordTok{plot}\NormalTok{(pressure, }\DataTypeTok{type =} \StringTok{'b'}\NormalTok{, }\DataTypeTok{pch =} \DecValTok{19}\NormalTok{)}
\end{Highlighting}
\end{Shaded}

\begin{figure}

{\centering \includegraphics[width=0.8\linewidth]{bookdown-demo_files/figure-latex/nice-fig-1} 

}

\caption{Here is a nice figure!}\label{fig:nice-fig}
\end{figure}

Reference a figure by its code chunk label with the \texttt{fig:}
prefix, e.g., see Figure \ref{fig:nice-fig}. Similarly, you can
reference tables generated from \texttt{knitr::kable()}, e.g., see Table
\ref{tab:nice-tab}.

\begin{Shaded}
\begin{Highlighting}[]
\NormalTok{knitr}\OperatorTok{::}\KeywordTok{kable}\NormalTok{(}
  \KeywordTok{head}\NormalTok{(iris, }\DecValTok{20}\NormalTok{), }\DataTypeTok{caption =} \StringTok{'Here is a nice table!'}\NormalTok{,}
  \DataTypeTok{booktabs =} \OtherTok{TRUE}
\NormalTok{)}
\end{Highlighting}
\end{Shaded}

\begin{table}

\caption{\label{tab:nice-tab}Here is a nice table!}
\centering
\begin{tabular}[t]{rrrrl}
\toprule
Sepal.Length & Sepal.Width & Petal.Length & Petal.Width & Species\\
\midrule
5.1 & 3.5 & 1.4 & 0.2 & setosa\\
4.9 & 3.0 & 1.4 & 0.2 & setosa\\
4.7 & 3.2 & 1.3 & 0.2 & setosa\\
4.6 & 3.1 & 1.5 & 0.2 & setosa\\
5.0 & 3.6 & 1.4 & 0.2 & setosa\\
\addlinespace
5.4 & 3.9 & 1.7 & 0.4 & setosa\\
4.6 & 3.4 & 1.4 & 0.3 & setosa\\
5.0 & 3.4 & 1.5 & 0.2 & setosa\\
4.4 & 2.9 & 1.4 & 0.2 & setosa\\
4.9 & 3.1 & 1.5 & 0.1 & setosa\\
\addlinespace
5.4 & 3.7 & 1.5 & 0.2 & setosa\\
4.8 & 3.4 & 1.6 & 0.2 & setosa\\
4.8 & 3.0 & 1.4 & 0.1 & setosa\\
4.3 & 3.0 & 1.1 & 0.1 & setosa\\
5.8 & 4.0 & 1.2 & 0.2 & setosa\\
\addlinespace
5.7 & 4.4 & 1.5 & 0.4 & setosa\\
5.4 & 3.9 & 1.3 & 0.4 & setosa\\
5.1 & 3.5 & 1.4 & 0.3 & setosa\\
5.7 & 3.8 & 1.7 & 0.3 & setosa\\
5.1 & 3.8 & 1.5 & 0.3 & setosa\\
\bottomrule
\end{tabular}
\end{table}

You can write citations, too. For example, we are using the
\textbf{bookdown} package \citep{R-bookdown} in this sample book, which
was built on top of R Markdown and \textbf{knitr} \citep{xie2015}.

\chapter{Literature}\label{literature}

Here is a review of existing methods.

\chapter{Methods}\label{methods}

We describe our methods in this chapter.

\chapter{Applications}\label{applications}

Some \emph{significant} applications are demonstrated in this chapter.

\section{Example one}\label{example-one}

\section{Example two}\label{example-two}

\chapter{Initial Data Tidying and exploration}\label{tidy_data}

\section{Introduction}\label{introduction}

\begin{quote}
`Can you do Addition?' the White Queen asked. `What's one and one and
one and one and one and one and one and one and one and one?' `I don't
know,' said Alice. `I lost count.'
\end{quote}

{[}REF TO THROUGH THE LOOKING GLASS{]}

Computer's operate with linear streams of discrete values. People don't.
One of the first challenges in data science is therefore understanding
how computers process digital data, when they are reading the data,
working with the data, and writing out the data. This chapter will
provide this understanding, in order to make sure you use approaches for
writing and reading data that are appropriate for the tasks.

\section{Eight ways of dividing file and data
types}\label{eight-ways-of-dividing-file-and-data-types}

There are many types of files on computers, usually identified with
different file extensions, such as \texttt{.zip}, \texttt{.csv},
\texttt{.exe}, \texttt{.doc}, \texttt{.pdf}, and so on. Such files can
be categorised into two groups:

\begin{itemize}
\tightlist
\item
  Text files
\item
  Binary files
\end{itemize}

A text file is a file that, when opened with a basic text editor, like
Notepad on Windows or TextEdit on OSX, will tend to display strings of
characters that are human readable. A poem or short note written in
Notepad would be an example of this, but so are other types of file
that, though not written so as to be read by people (at least not for
fun or enlightenment) nevertheless clearly contain text when displayed
in a text editor.

As an example of a text file written in a plain text format, we can look
to Project Gutenberg {[}REF{]}, an online repository of classic
literature that's old enough to be copyright free. This is accessible
within R through a package called \texttt{gutenbergr}{[}REF{]}. With a
few lines of code we can download and jump right into some plain text
data clearly meant for humans rather than machines:

\begin{Shaded}
\begin{Highlighting}[]
\KeywordTok{require}\NormalTok{(tidyverse)}
\KeywordTok{require}\NormalTok{(gutenbergr)}
\NormalTok{gutenberg_metadata }\OperatorTok\StringTok{ }
\StringTok{  }\KeywordTok{filter}\NormalTok{(title }\OperatorTok{==}\StringTok{ "Through the Looking-Glass"}\NormalTok{)}
\end{Highlighting}
\end{Shaded}

\begin{verbatim}
## # A tibble: 2 x 8
##   gutenberg_id title   author  gutenberg_autho~ language gutenberg_booksh~
##          <int> <chr>   <chr>              <int> <chr>    <chr>            
## 1           12 Throug~ Carrol~                7 en       Children's Liter~
## 2        23718 Throug~ Carrol~                7 en       Children's Liter~
## # ... with 2 more variables: rights <chr>, has_text <lgl>
\end{verbatim}

\begin{Shaded}
\begin{Highlighting}[]
\NormalTok{looking_glass <-}\StringTok{ }\KeywordTok{gutenberg_download}\NormalTok{(}\DecValTok{12}\NormalTok{)}

\NormalTok{looking_glass[}\DecValTok{100}\OperatorTok{:}\DecValTok{105}\NormalTok{,}\StringTok{"text"}\NormalTok{]}
\end{Highlighting}
\end{Shaded}

\begin{verbatim}
## # A tibble: 6 x 1
##   text                                                                    
##   <chr>                                                                   
## 1 that nasty Knight, that came wiggling down among my pieces. Kitty, dear,
## 2 let's pretend--' And here I wish I could tell you half the things Alice 
## 3 used to say, beginning with her favourite phrase 'Let's pretend.' She   
## 4 had had quite a long argument with her sister only the day before--all  
## 5 because Alice had begun with 'Let's pretend we're kings and queens;' and
## 6 her sister, who liked being very exact, had argued that they couldn't,
\end{verbatim}

By contrast, binary files are all other file types. The term `binary'
refers to the fact that computers work with zeros and ones. Binary files
are those that don't clearly or obviously display as text when you look
at them through a basic text editor. Instead the computer needs
something more specialised to the type of data in order to make sense of
it.

The \texttt{readLines} function tries to load and display the contents
of any file, assuming it is a text file, even when it is not. For
example, if we try to read the first five `lines' of an .mp3 file of a
song by the Norwegian electronic music duo Royksopp, we get the
following:

\begin{Shaded}
\begin{Highlighting}[]
\KeywordTok{readLines}\NormalTok{(}\StringTok{"chapter_support/05/03 49 Percent.mp3"}\NormalTok{, }\DataTypeTok{n =} \DecValTok{5}\NormalTok{)}
\end{Highlighting}
\end{Shaded}

\begin{verbatim}
## Warning in readLines("chapter_support/05/03 49 Percent.mp3", n = 5): line 1
## appears to contain an embedded nul
\end{verbatim}

\begin{verbatim}
## Warning in readLines("chapter_support/05/03 49 Percent.mp3", n = 5):
## incomplete final line found on 'chapter_support/05/03 49 Percent.mp3'
\end{verbatim}

\begin{verbatim}
## [1] "ID3\003"                                                                                                                                                                  
## [2] "´ÙUYY@¥«½°7Ž;\210\027\031\177\200ÔHÅ4m\025½uÅË H\vè'\023¨\\6y\036+\035\026Ã!|Ós\200œPA4\025\020Óê6Ä2#õdL«Üà+\034!«\t\vŠÍ”\024\030\\0aq[j"                                           
## [3] "\030\\PÂâ¶Ô\0240¸¡…ÉÞ¡#\v’0º=Q\003—$rèõD\030º\f]\036¨ƒ\027A‹£Õ\020dÐbèõF2h1võF2lEvä£\036lbíÒŒdØŠó¥!sc\027G$\b\"\215\210£n\2201\024lZ6é\003\026\215‹FÜ\020BÛ`\\qÂãOèYðÁ5¿$Nï”/\020\027"
\end{verbatim}

(Note: the above text still does not make any sense even if you read
Norwegian!)

A second way of distinguishing between types of files, more relevant
when thinking about data files, is the following:

\begin{itemize}
\tightlist
\item
  Rectangular data files
\item
  Non-rectangular data files
\end{itemize}

To some extent, the difference between these two types of data should be
obvious: a rectangular data file contains data that the computer expects
to be arranged in some kind of rectangular table. Non-rectangular data
is any other type of data.

An example of rectangular data, available from the data-sharing website
\texttt{figshare}, is shown below. The data reports the body mass index
(BMIs) of participants in a study recorded at different time points. In
the example below, the first five lines of the data are first loaded
with the \texttt{readLines} function we used earlier, then using a
special function, \texttt{read\_csv} designed to work with rectangular
data through an understanding of how the stream of data inside the file
is split into different columns and rows:

\begin{Shaded}
\begin{Highlighting}[]
\KeywordTok{readLines}\NormalTok{(}\StringTok{"https://ndownloader.figshare.com/files/10905845"}\NormalTok{, }\DataTypeTok{n =} \DecValTok{5}\NormalTok{)}
\end{Highlighting}
\end{Shaded}

\begin{verbatim}
## [1] "2011,30.684,24.894,39,0"    "2012,24.894,-1,40,0"       
## [3] "2003,20.751,22.991,26,1"    "2004,22.991,23.345,27.25,1"
## [5] "2005,23.345,22.828,28,1"
\end{verbatim}

\begin{Shaded}
\begin{Highlighting}[]
\KeywordTok{read_csv}\NormalTok{(}\StringTok{"https://ndownloader.figshare.com/files/10905845"}\NormalTok{, }\DataTypeTok{n_max =} \DecValTok{5}\NormalTok{)}
\end{Highlighting}
\end{Shaded}

\begin{verbatim}
## Parsed with column specification:
## cols(
##   `2011` = col_integer(),
##   `30.684` = col_double(),
##   `24.894` = col_double(),
##   `39` = col_double(),
##   `0` = col_integer()
## )
\end{verbatim}

\begin{verbatim}
## # A tibble: 5 x 5
##   `2011` `30.684` `24.894`  `39`   `0`
##    <int>    <dbl>    <dbl> <dbl> <int>
## 1   2012     24.9     -1    40       0
## 2   2003     20.8     23.0  26       1
## 3   2004     23.0     23.3  27.2     1
## 4   2005     23.3     22.8  28       1
## 5   2006     22.8     22.1  29       1
\end{verbatim}

We can see in the two examples above that, though the \texttt{readLines}
function has split the above data into different rows, the contents
within each row are stuck together, separated by the \texttt{,} symbol.
The \texttt{read\_csv} function, however, has split these values into
separate columns, producing the rectangle of data that we largely
expect. The \texttt{read\_csv} output still has some problems, which we
will fix later, but it's more suitable for this type of data.

It's likely that you'll be working with rectangular data most of the
time. However it's useful to be aware of what non-rectangular data can
look like, and to think about the cases where non-rectangular data
structures are most appropriate. Much of the internet is built using
structured non-rectangular data written as text files. An simple example
of this is the following:

\begin{Shaded}
\begin{Highlighting}[]
\KeywordTok{require}\NormalTok{(jsonlite)}

\KeywordTok{readLines}\NormalTok{(}\StringTok{"chapter_support/05/simple_json.json"}\NormalTok{)}
\end{Highlighting}
\end{Shaded}

\begin{verbatim}
##  [1] "{"                                  
##  [2] "  \"edibles\": {"                   
##  [3] "    \"fruit\": {"                   
##  [4] "      \"name\":\"apples\", "        
##  [5] "      \"name\":\"oranges\", "       
##  [6] "      \"name\":\"tomatoes\", "      
##  [7] "      \"name\":\"lemons\""          
##  [8] "     },"                            
##  [9] "    \"vegetables\": {"              
## [10] "      \"name\":\"potatoes\", "      
## [11] "      \"name\":\"sweet potatoes\", "
## [12] "      \"name\":\"aubergines\""      
## [13] "    }"                              
## [14] "  },"                               
## [15] "  \"inedibles\": {"                 
## [16] "    \"name\":\"tables\", "          
## [17] "    \"name\":\"benches\", "         
## [18] "    \"name\":\"chairs\", "          
## [19] "    \"name\":\"pots\""              
## [20] "  }"                                
## [21] "}"
\end{verbatim}

\begin{Shaded}
\begin{Highlighting}[]
\KeywordTok{read_json}\NormalTok{(}\StringTok{"chapter_support/05/simple_json.json"}\NormalTok{)}
\end{Highlighting}
\end{Shaded}

\begin{verbatim}
## $edibles
## $edibles$fruit
## $edibles$fruit$name
## [1] "apples"
## 
## $edibles$fruit$name
## [1] "oranges"
## 
## $edibles$fruit$name
## [1] "tomatoes"
## 
## $edibles$fruit$name
## [1] "lemons"
## 
## 
## $edibles$vegetables
## $edibles$vegetables$name
## [1] "potatoes"
## 
## $edibles$vegetables$name
## [1] "sweet potatoes"
## 
## $edibles$vegetables$name
## [1] "aubergines"
## 
## 
## 
## $inedibles
## $inedibles$name
## [1] "tables"
## 
## $inedibles$name
## [1] "benches"
## 
## $inedibles$name
## [1] "chairs"
## 
## $inedibles$name
## [1] "pots"
\end{verbatim}

Non-rectangular data structures are hierarchical and tree-like. For
example, the structure of the above can be shown graphically as follows:

\begin{Shaded}
\begin{Highlighting}[]
\NormalTok{json_dta <-}\StringTok{ }\KeywordTok{read_json}\NormalTok{(}\StringTok{"chapter_support/05/simple_json.json"}\NormalTok{)}
\NormalTok{json_dta }\OperatorTok\StringTok{ }\NormalTok{data.tree}\OperatorTok{::}\KeywordTok{as.Node}\NormalTok{() }
\end{Highlighting}
\end{Shaded}

\begin{verbatim}
##          levelName
## 1 Root            
## 2  ¦--edibles     
## 3  ¦   ¦--apples  
## 4  ¦   °--potatoes
## 5  °--tables
\end{verbatim}

Part (a) will introduce the primary distinction used by computer
programmers between types of files, between human-readable text files
and human-unreadable binary files. Examples of these two different types
of files will be introduced, which also highlight another important
conceptual distinction: the difference between data and meta-data (data
about data). The relative advantages of different file types will be
discussed, and the argument will be made that, as a rule-of-thumb, text
files should be used in preference to binary files.

\section{Loading and saving data - R packages and
functions}\label{loading-and-saving-data---r-packages-and-functions}

Part (b) will discuss some of the particular R packages and functions
within that can be used to work with different file types, presenting
even the loading of data as at iterative process of learning more about
how variables in data are structured and saved, and how the information
in these variables be best represented and operated with inside R.

\section{Data tidying - principles and
challenges}\label{data-tidying---principles-and-challenges}

Part (c) will introduce a further, higher level, distinction between
types of data: from `untidy' and `initial' data to `tidy' and `derived'
data. This section will frame much of the practical challenge of data
science as involving first identifying what needs to be done to move
from the former to the latter, and then knowing how to do this.

\section{Data tidying - tidyr and
dplyr}\label{data-tidying---tidyr-and-dplyr}

Part (d) will introduce two closely linked R packages, tidyr and dplyr,
as providing the tools necessary to complete the vast majority of data
tidying and data derivation tasks.

\section{Initial Exploratory data
analysis}\label{initial-exploratory-data-analysis}

Finally, part (e) will emphasise the importance of rapid exploratory
data analysis both at the data tidying stage, and for developing
familiarity and engagement with tidied data sources.

\chapter{Data Visualisation}\label{data_vis}

\section{What is data visualisation? The grammar of graphics
approach}\label{what-is-data-visualisation-the-grammar-of-graphics-approach}

Part (a) will introduce Leland Wilkinson's `grammar of graphics'
philosophy to thinking about data visualisation, and for producing clear
modular instructions for the production of data graphics.

\section{Exploratory data analysis using
ggplot2}\label{exploratory-data-analysis-using-ggplot2}

Part (b) will describe how the concept of the `grammar of graphics' has
been operationalised within R within the hugely popular ggplot2 package,
and more recently ggvis, and present some extended examples of their use
in exploring the tidied dataset introduced previously.

\section{Producing tables in R}\label{producing-tables-in-r}

Part (c) will cover the considerations and practicalities of table
production, both in the context of exploratory data analysis, and when
producing final outputs for papers and reports. The relative merits of
figures and tables -- both in principle and in practice -- will also be
discussed at this stage.

\section{Producing publication-ready
graphs}\label{producing-publication-ready-graphs}

Part (d) will return to more conventional types of data visualisation,
and address how to polish and tweak the graphs produced using ggplot2
and related packages into something publication ready. Issues such as
how to control cosmetic features such as background styles, fonts and
tick label orientation, will be discussed, as well as how to produce
both monochrome and colour versions of images, how to add additional
labels and captions within figures, and how to save images as files of
the correct type and resolution.

\section{An introduction to spatial
maps}\label{an-introduction-to-spatial-maps}

Part (e) will provide an introduction to using R to produce maps.
Spatial data are handled very differently in R to most other types of
data featured in this book, and so working with such data produces a
series of additional challenges. However, maps can be immensely powerful
tools for analysis, often revealing otherwise hidden spatial structures
to social data. Although the challenges of working with spatial data
mean that many of the packages introduced earlier in the book are not
appropriate, this part of the chapter will emphasise continuity with
previous material on data visualisation by focusing on two packages
which are both inspired by the grammar of graphics philosophy, ggmap and
tmap.

\chapter{Text analysis}\label{text_explore}

This section will provide a brief introduction to using R to explore
text documents, an unusual type of data often encountered more
associated with qualitative than quantitative research in the social
sciences. This chapter will not provide an extensive, formal discussion
of how to produce statistical models based on text, but instead should
serve as a taster of some of the features of the recent
\texttt{tidytext} package by XXX.

\chapter{Statistics}\label{stats}

Most books on quantitative methods focus very heavily on statistics.
This book will instead situate statistics as an element within a much
broader process of activities in the service of getting insight from
data, and making better decisions on the basis of this insight.

\section{The aims of statistics: data reduction or
inference?}\label{the-aims-of-statistics-data-reduction-or-inference}

Part (a) will argue for statistics, by which is meant primarily the
production of summary statistics, as a key part of the data science
toolkit from two parallel perspectives: in addition to recourse to
theories of statistical inference, which have as their key concern the
appropriate use of data to populate parameters in stochastic models,
there are also important cognitive reasons for producing summary
statistics, namely that often more data means less information, as
people are severely limited both in time and memory as to the number of
`facts' they can hold onto and make sense of at a time. Summary
statistics are therefore invaluable as tools for intelligent data
reduction: data reduction because it turns many thousands or millions of
observations into just a handful of key values, and intelligent because
these key values are `less worse' at standing in place of many more
values than most alternatives. The relationship between statistical
modelling and signal processing will also be discussed. Theories of
statistical inference, and in particular likelihood theory, provide a
means of selecting between many different models -- each assuming
different stochastic structures and processes, and hence each embodying
different theories about social phenomena -- and of deciding on the
right balance between model accuracy and model parsimony, but only when
models have a particular structural relationship to each other (When one
model can be thought of as a `constrained' variant of another model).

\section{Balancing accuracy and parsimony using AIC and
BIC}\label{balancing-accuracy-and-parsimony-using-aic-and-bic}

Part (b) will discuss this further, and offer Aikiake Information
Criterion (AIC) and the Bayesian Information Criterion (BIC) as a way of
guiding the accuracy/parsimony trade-off in selecting between many types
of statistical model.

\section{Statistical Simulation: Showing what models really
mean}\label{statistical-simulation-showing-what-models-really-mean}

Having decided on the `least worst' model to represent the data, part
(c) will describe a series of methods and functions -- including sim
within the arm package and the Zelig package -- for extracting
substantive meaning from statistical models, going beyond the usual
convention of stopping and starting with the question of whether
particular model parameters are `statistically significant' or not, to
instead making predictions and projections about the expected and
predicted effects of change within social systems.

\section{Advanced statistics: structural modelling to formalise informal
knowledge}\label{advanced-statistics-structural-modelling-to-formalise-informal-knowledge}

Part (d) will discuss further how statistical models can be developed to
formalise implicit and `fuzzy' knowledge and expectations about social
relationships, discussing both the dangers of going `too far' and
developing such complicated and precisely structured models that it
becomes easy to confuse `the map for the territory', and the danger of
not going far enough and failing to capture important structure and
shape in the relationships between variables and social factors.

\chapter{Programming and Automation}\label{Coding}

A guiding principle in this book is the following: more emphasis should
be given to training students about technical issues in quantitative
methods courses, because technical issues are not interesting. This is
because often `technical issues', such as ensuring software are loaded
correctly and searching for bugs and syntax errors in code, can take up
a large proportion of the time available to social researchers to do
research; if the definition of `technical issues' were expanded to
include many routine `janitorial' data management issues like
reformatting data and managing different versions of code and documents,
then `technical issues' usually take the majority of researchers' time.
The rationale for focusing more on teaching methods and approaches for
handling technical issues is therefore that if even small efficiency
gains can be found in some of these areas then much more time will be
freed to perform more interesting and substantively meaningful tasks,
such as exploring data and thinking carefully about interpretations,
implications and social hypotheses.

R is a statistical programming language rather than simply a statistical
package. What this means is that R has a steeper learning curve
initially than a statistical package, but that both the range and
efficiency of data science tasks which can be performed in R can be much
greater than for a standard statistical software package. However, in
order to make use of the benefits of R, it needs to be approached as a
programming language, and in order to do this, something about
programming languages needs to be covered within this book. This chapter
will provide this necessary background knowledge required to make best
use of R in data science, and in particular for using its functionality
to greatly cut down the amount of time and effort required to solve
routine technical and janitorial data science tasks.

\section{Writing user-defined
functions}\label{writing-user-defined-functions}

Functions will have been used throughout the book up until this point.
Part (a) formally introduces them in R, providing explicit guidance on
what functions are, how they work, and how they can be used to speed up,
standardise, and reduce the scope for errors to accumulate within the
data-to-value chain. Functions provide building blocks for automation,
and automation means that the marginal costs of doing more, in terms of
researcher time and effort, fall to almost nothing, and so more data
analyses and explorations can be performed.

\section{Automating analyses using
purrr}\label{automating-analyses-using-purrr}

Part (b) moves from the individual function, to functional programming
philosophies more generally, and shows how this broader understanding of
functional programming allow for automation of tasks at each and every
stage of the data-to-value process. The purrr package, effectively a
series of functions for working with functions, will be introduced at
this stage; purrr functions integrate with user-defined functions to
allow the same complex processes and procedures to be applied many times
to many different pieces or chunks of data.

\section{Automating data tidying}\label{automating-data-tidying}

Parts (c), (d) and (e) will provide examples of purrr in use for a range
of different activities: automating the loading, tidying and arranging
of data from many different files; automating the production of graphs
and summary findings from many different sections of a dataset,
producing dozens or hundreds of files at a time; and automating either
the running of a single statistical model to multiple pieces of data, or
alternatively running many different types of model to a single dataset

\section{Automating graphs and
tables}\label{automating-graphs-and-tables}

Parts (c), (d) and (e) will provide examples of purrr in use for a range
of different activities: automating the loading, tidying and arranging
of data from many different files; automating the production of graphs
and summary findings from many different sections of a dataset,
producing dozens or hundreds of files at a time; and automating either
the running of a single statistical model to multiple pieces of data, or
alternatively running many different types of model to a single dataset

\section{Automating statistical
analyses}\label{automating-statistical-analyses}

Parts (c), (d) and (e) will provide examples of purrr in use for a range
of different activities: automating the loading, tidying and arranging
of data from many different files; automating the production of graphs
and summary findings from many different sections of a dataset,
producing dozens or hundreds of files at a time; and automating either
the running of a single statistical model to multiple pieces of data, or
alternatively running many different types of model to a single dataset

.

\chapter{Open, Reproducible and Collaborative
Research}\label{repro_research}

A vitally important aspect of any science is to be able to produce
claims that can be tested and verified by other people. As research
activities are tending to become more complex - conceptually,
methodologically and technically -- there need to work collaboratively
has become increasingly important.

\section{The theory and challenges of open and reproducible
research}\label{the-theory-and-challenges-of-open-and-reproducible-research}

Part (a) will consider both the benefits and the barriers to open and
reproducible research in the context of the social and health sciences,
where both the nature of the data and the background of researchers tend
to be very different to disciplines like physics, where open and
reproducible research practices tend to be more commonplace.

\section{Sharing results and findings using
github}\label{sharing-results-and-findings-using-github}

Part (b) will turn to the practical means and benefits of using GitHub,
integrated with Rstudio projects, for making both the code and processes
produced available to your future self, but also tidied and derived
data, and initial findings and figures, available to other researchers.

\section{Writing collaboratively with wikis, markdown and related
approaches}\label{writing-collaboratively-with-wikis-markdown-and-related-approaches}

Part (c) will discuss linked features and programs that allow draft
reports and other written material to be shared and co-developed with
other researchers.

\section{Managing literature
(Optional)}\label{managing-literature-optional}

Finally part (d) covers approaches to managing literature and literature
review in the context of R projects and the research-project-as-suitcase
philosophy advocated throughout the book.

\bibliography{book.bib,packages.bib}


\end{document}
